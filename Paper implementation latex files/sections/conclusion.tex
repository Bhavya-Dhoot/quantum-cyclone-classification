%%% ============================================================
%%% SECTION 7: CONCLUSION
%%% ============================================================
\section{Conclusion}\label{sec:conclusion}

Using a six-qubit IQP feature map and a 600-sample training subset of IBTrACS data, the proposed quantum kernel SVM achieved 96.1\% classification accuracy. We investigated whether the explicit computational hardness properties inherent within IQP circuits supply functional analytical machine learning advantages natively. Experimental output processing yielded a restricted, positive conclusion structurally. We formulated a distinct operational quantum kernel model utilising parameterised single-layer IQP hardware circuits. The developed architecture encoded discrete specific meteorological values strictly alongside explicit pairwise interacting product calculations. The executed model recorded exactly 96.1\% diagnostic global accuracy categorising three-class cyclone storm intensity levels. Hardware testing occurred strictly against a limited 600-sample balanced partition extracted from the public IBTrACS global archive.

Classical baselines consumed ten times the raw data volume and hit near-perfect separation computational boundaries immediately. The original six atmospheric input features provided explicit fundamental linear separability strictly under massive high-data processing conditions. However, the IQP mathematical architecture completely dominated competing entangled quantum benchmarks under identical data restrictions. The functional IQP method beat a generic fully entangled ZZFeatureMap quantum SVM by exactly 12.2 isolated percentage points. This statistical separation measurement manifested primarily within the highly complex moderate hurricane boundary category. Physical atmospheric variables mapped along the tropical storm intensity boundary interact through highly complex nonlinear physical mechanics continuously. Specific designed structural circuit topologies parse these exact mechanics significantly more efficiently than unstructured generic dense entanglement mapping arrays.

Circuit quantum depth analysis testing isolated $L = 1$ inherently as the explicit mathematical processing optimum limit. Extending quantum circuit logical depth boundaries steadily degraded aggregate final classification diagnostic accuracy linearly. This exact empirical functional decay structurally matches formal algorithmic theoretical expectations regarding flat kernel data matrix concentration limits mapped inside hyper-expressive processing spaces perfectly. Single-layer baseline IQP encodings achieved the absolute top geometric structural label fit mathematical score calculated among all tested entangled operational maps natively.

\subsection{Future Work}

Future research pathways isolate strictly into six primary operational implementation domains:

\begin{enumerate}
    \item \textbf{Execution on quantum hardware.} Researchers must deploy this exact IQP mathematical formulation directly onto physical physical NISQ processing devices. Natural computational gate errors and finite physical coherence tracking spans will immediately compromise kernel geometry measurements natively. Systematic operational hardware testing will explicitly quantify this exact physical degradation slope mathematically. Protocol software pipelines will absolutely require integrated algorithmic zero-noise extrapolation mitigation loops.

    \item \textbf{Feature map optimisation.} We arbitrarily anchored the exact internal phase scaling variables $\alpha_j$ and $\beta_{jk}$ permanently at base unity values initially. Implementing formal kernel-target data alignment algorithmic maximisation loops through short iterative variational processing bounds will dynamically optimise these specific variables structurally. Advanced internal network computational graphs could theoretically learn the exact physical qubit hardware connectivity routing patterns structurally mirroring strong physical atmospheric tracking couplings explicitly.

    \item \textbf{Richer feature sets.} A completely basic six-variable model represents only fundamental core atmospheric physics inherently. Advanced global reanalysis tracking datasets provide vast complex variable integration processing options natively. Sea surface oceanic baseline temperature data vectors, CAPE vertical profile limits, relative ambient humidity horizontal gradients, and total measurable oceanic heat tracking content values directly physically govern ongoing vortex intensification mechanics. Functionally incorporating fifteen operational predictive variables demands massively expanded physical qubit hardware topologies or destructive hybrid classical--quantum dimensional bottleneck restrictions structurally.

    \item \textbf{Temporal structure.} Tropical cyclone physical intensity vectors represent continuous chronological time series functions structurally, absolutely not static isolated single-point atmospheric snapshots. Connecting the deployed baseline static IQP kernel processor to a functional sequential recursive architecture remains a completely unsolved computational problem. A specific quantum mathematical mapping equivalent constructing a hidden Markov probability model or functioning recurrent sequential logic frame will actually capture physical storm intensification momentum dynamics entirely mathematically overlooked by instantaneous static logic decision planes currently.

    \item \textbf{Operational deployment.} Real-time physical weather forecasting structurally enforces incredibly strict processing latency mathematical barriers. A functional diagnostic classifier architecture algorithm demanding 47~minutes per single discrete kernel computation explicitly fails standard three-hour global broadcast update cycles natively. Physical operational quantum processing hardware completely alters these basic latency interaction dynamics entirely. The backend digital infrastructure functionally necessary to seamlessly integrate physical hardware QPU execution subroutines directly into daily rigid weather computation pipelines currently does absolutely not exist globally.
    
    \item \textbf{Generalisation to other phenomena.} Tropical cyclones represent strictly a singular localized vector tracked within massive global computational atmospheric classification network architectures. Structural tornado formation physical detection models, global monsoon boundary tracking crossing vectors, and complex extratropical transition decay routing functions rely explicitly on heavy complex nonlinear physical variable measurement interaction tracking limits. Directly evaluating the functional IQP algorithmic framework systematically across these highly divergent independent tracking domains will mathematically verify explicitly if the recorded baseline accuracy processing jump stems solely from specific isolated cyclone physics mechanics natively or represents a foundational general tracking encoding geometry performance advantage structurally.
\end{enumerate}

Quantum processor hardware currently continues mathematically progressing directly down standard established operational scaling metric curves aggressively. Total functional qubit physical volumes continuously expand, component gate error probability rates mathematically decline steadily, and native hardware gate operational speed execution architectures improve linearly. Continuous global hardware improvements will inevitably eventually cleanly override the baseline software simulation processor bottlenecks currently mathematically isolated directly inside this specific research paper completely. Formal quantum classification kernel algorithm formulations anchored directly to mathematically explicitly robust underlying computational problem hardness properties structurally supply a highly clear, explicitly defined functional vector specifically targeted for advancing overall global atmospheric predictive machine learning tracking protocols massively.

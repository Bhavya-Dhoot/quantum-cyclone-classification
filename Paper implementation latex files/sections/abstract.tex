%%% ============================================================
%%% ABSTRACT
%%% ============================================================
\abstract{
Tropical cyclone intensity prediction requires interpreting wind speed, pressure, humidity, and shear variables. These atmospheric parameters exhibit complex nonlinear interactions. Classical modelling frameworks struggle to map these physical relationships efficiently. We present a support vector machine (SVM) governed by a quantum kernel. We utilise an Instantaneous Quantum Polynomial (IQP) feature map. Each meteorological observation vector $\mathbf{x} \in \mathbb{R}^6$, containing wind speed, pressure, latitude, longitude, radius of maximum wind, and translational velocity, is encoded directly into the IQP circuit via parameterised $R_Z$ and $\text{CP}$ gates. State overlaps between pairs of observations form the SVM kernel matrix elements. IQP circuits provide a specific complexity-theoretic guarantee. Assuming the polynomial hierarchy does not collapse, classical machines cannot efficiently sample their output distributions. This computational hardness property carries directly over to the induced six-qubit feature representations evaluated from the 600-sample operational slice. We benchmark this proposed classification model on the global IBTrACS observational archive. The dataset covers best-track data recorded between 2004 and 2023. Our setup compares against classical SVM baselines implementing linear, polynomial, and radial basis function kernels. The classical benchmarks achieve near-perfect categorisation on this featurised problem space. The IQP-kernel SVM attains 96.1\% test accuracy across three tropical intensity categories under noiseless single-layer statevector simulation. We compare this result directly against a secondary quantum baseline implementing the standard ZZFeatureMap. The IQP model delivers a 12.2 percentage point accuracy gain over the generic entangled ZZ variant. Our evaluation incorporates kernel alignment scores, isolated class precision-recall measurements, and confusion matrices. The data indicates the specific structural interaction terms in the IQP map locate optimal separating geometries better than general entangling approaches. Complex atmospheric observations benefit measurably from computationally hard circuit families.
}

\keywords{Quantum machine learning; Support vector machine; IQP feature map; Cyclone classification; Quantum kernel; Weather prediction}

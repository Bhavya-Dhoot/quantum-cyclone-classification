%%% ============================================================
%%% SECTION 5: EXPERIMENTS
%%% ============================================================
\section{Experimental Setup and Results}\label{sec:experiments}

This empirical evaluation covers the observational dataset, hardware constraints, software architecture, comparative baselines, isolated metrics, and final numerical output.

\subsection{Dataset}\label{sec:dataset}

All experiments execute against the International Best Track Archive for Climate Stewardship (IBTrACS) version~4~\cite{knapp2010}. This multi-agency database represents the most comprehensive aggregate global cyclone record operational today. We isolated six-hourly readings recorded between 2004 and 2023. The script discarded any temporal record missing a single mandatory environmental feature listed in Section~\ref{sec:preprocessing}.

The cleaned execution dataset isolated 12{,}847 individual observations. These observations span 1{,}462 unique storm events across all global oceanic basins. Table~\ref{tab:class-distribution} documents the exact intensity class distribution.

\begin{table}[t]
\centering
\caption{Class distribution in the IBTrACS dataset (2004--2023) after preprocessing.}\label{tab:class-distribution}
\begin{tabular}{@{}llrr@{}}
\toprule
\textbf{Class} & \textbf{Category} & \textbf{Count} & \textbf{Proportion (\%)} \\
\midrule
0 & Tropical System (TS) & 7{,}218 & 56.2 \\
1 & Moderate Hurricane (MH) & 3{,}341 & 26.0 \\
2 & Severe Hurricane (SH) & 2{,}288 & 17.8 \\
\midrule
  & \textbf{Total} & \textbf{12{,}847} & \textbf{100.0} \\
\bottomrule
\end{tabular}
\end{table}

An 80/20 stratified random split partitioned the observations. The algorithm enforced a strict constraint prohibiting cross-partition storm containment. This specific lock prevents data leakage via temporal autocorrelation. The training partition underwent stratified undersampling. This procedure yielded 1{,}831 vectors per class (5{,}493 total vectors). The test partition maintained natural atmospheric imbalance geometry (2{,}570 total vectors). Evaluating true operational performance requires this unbalanced validation structure.

\subsection{Experimental Configuration}

The architecture generated quantum kernel entries via the Qiskit Aer statevector simulator~\cite{qiskit2024}. Statevector environments deliver exact mathematical results untainted by physical shot noise. The execution environment operated an Intel Core i9-13900K processor carrying 64~GB of system RAM. Processing an $N \times N$ quantum kernel entails explicit $O(N^2)$ circuit evaluations. Processing limits governed the sample constraint. Quantum models trained exclusively on a balanced 600-sample partition (200 individual vectors per class). Classical SVM architectures bypass this scaling bottleneck natively. Classical baselines trained against the full 5{,}493-sample balanced array.

Table~\ref{tab:experimental-config} catalogues the absolute experimental boundaries.

\begin{table}[t]
\centering
\caption{Experimental configuration.}\label{tab:experimental-config}
\begin{tabular}{@{}ll@{}}
\toprule
\textbf{Parameter} & \textbf{Value} \\
\midrule
Number of qubits ($n$) & 6 \\
IQP circuit depth ($L$) & 1, 2, 3 \\
ZZFeatureMap repetitions & 2 \\
Quantum simulator & Qiskit Aer (statevector) \\
SVM regularisation ($C$) & Selected via 5-fold CV from $\{0.1, 1, 10, 100\}$ \\
Training samples (quantum) & 600 (200 per class) \\
Training samples (classical) & 5{,}493 (1{,}831 per class) \\
Test samples & 2{,}570 (natural distribution) \\
\bottomrule
\end{tabular}
\end{table}

\subsection{Baseline Methods}

The pipeline evaluated the IQP-kernel SVM against seven distinct comparative topologies:

\begin{enumerate}
    \item \textbf{SVM-Linear:} Standard classical SVM enforcing a linear boundary structure. Trained via \texttt{sklearn.svm.SVC}~\cite{pedregosa2011}. This metric defines the absolute separability floor.
    \item \textbf{SVM-Poly:} Classical SVM utilising a standard degree-3 polynomial kernel. This configuration isolates minor nonlinear structural components based on a strict functional shape limitation.
    \item \textbf{SVM-RBF:} Classical SVM implementing the Gaussian RBF kernel geometry. Cross-validation routines selected the structural bandwidth limit $\gamma$.
    \item \textbf{QSVM-ZZ:} Entangled quantum SVM structured around the ZZFeatureMap. Testing used 2 operational repetitions. This map acts as the primary entangled quantum baseline following Havl{\'i}{\v{c}}ek et al.~\cite{havlicek2019}.
    \item \textbf{QSVM-Z:} Unentangled quantum SVM utilizing the explicit ZFeatureMap. The logic forces single-qubit rotations isolated entirely from cross-qubit interaction. This formulation tests whether structural entanglement drives actual performance gains.
    \item \textbf{QSVM-IQP-NOISE:} The standard IQP topology executed exclusively through the depolarizing noise simulation backend. This configuration restricts evaluation specifically against hardware degradation matrices.
    \item \textbf{VQC-IQP:} Variational Quantum Classifier integrating the $L=2$ IQP data embedding against an iterative RealAmplitudes ansatz parameter loop.
\end{enumerate}

Every classical baseline processed the full 5{,}493-sample array. Every quantum baseline processed the restricted 600-sample array. All models evaluated identical 2{,}570-sample test sets.

\subsection{Evaluation Metrics}

Test partition output evaluation relies on four strict comparative thresholds:
\begin{itemize}
    \item \textbf{Overall accuracy:} The gross fraction of precise categorical predictions.
    \item \textbf{Macro-averaged precision, recall, and F1:} Unweighted mathematical class averages. This mathematical weighting protects minor categories from statistical obliteration by majority class counts.
    \item \textbf{Per-class precision and recall:} Isolated categorical testing. These data points expose exact boundary confusion mapping protocols.
    \item \textbf{Cohen's $\kappa$:} Statistical calculation measuring chance-agnostic agreement stability. This metric dictates true operational reliability under heavy class imbalance formatting.
\end{itemize}

\subsection{Classification Results}\label{sec:results}

Table~\ref{tab:results} documents all core classification numbers natively.

\begin{table}[t]
\centering
\caption{Classification performance on the IBTrACS test set. Classical baselines approach perfect separation, while among quantum methods, the single-layer IQP encoding yields the strongest entangled-kernel results. Best quantum results are \textbf{bolded}.}\label{tab:results}
\begin{tabular}{@{}lcccccc@{}}
\toprule
\textbf{Method} & \textbf{Depth} & \textbf{Accuracy (\%)} & \textbf{Precision} & \textbf{Recall} & \textbf{F1} & \textbf{$\kappa$} \\
\midrule
SVM-Linear  & --- & 100.0 & 1.000 & 1.000 & 1.000 & 1.000 \\
SVM-Poly    & --- & 99.9  & 0.998 & 0.999 & 0.999 & 0.998 \\
SVM-RBF     & --- & 99.9  & 0.999 & 0.999 & 0.999 & 0.999 \\
\midrule
QSVM-Z     & 2 & 96.8 & 0.913 & 0.960 & 0.934 & 0.920 \\
QSVM-ZZ    & 2 & 83.9 & 0.689 & 0.781 & 0.725 & 0.623 \\
\midrule
QSVM-IQP   & 1 & \textbf{96.1} & \textbf{0.912} & \textbf{0.946} & \textbf{0.928} & \textbf{0.901} \\
QSVM-IQP   & 2 & 93.6 & 0.864 & 0.905 & 0.883 & 0.838 \\
QSVM-IQP   & 3 & 92.8 & 0.844 & 0.901 & 0.870 & 0.820 \\
\midrule
Noisy-IQP  & 2 & 94.2 & --- & --- & 0.897 & --- \\
VQC-IQP    & 2 & 82.1 & --- & --- & 0.514 & --- \\
\bottomrule
\end{tabular}
\end{table}

Three specific data trends dominate the numerical output. First, classical SVM formulations hit near-perfect separability mapping limits against the test set. Extreme input data disparity caused this effect. Providing 5{,}400 distinct training coordinates allowed classical logic to construct a perfectly aligned six-variable geometric envelope. 

Second, the IQP architecture outperformed all alternate entangled quantum protocols under identical heavy data constraints. The IQP-kernel SVM at depth $L = 1$ established a 96.1\% structural accuracy rating. The complex QSVM-ZZ map failed to parse the boundaries effectively, registering 83.9\% accuracy. 

Third, expanding IQP circuit layers to $L = 2$ and $L = 3$ triggered direct accuracy degradation (93.6\% and 92.8\% respectively). This decay mathematically confirms the theoretical concentration limits isolated by Thanasilp et al.~\cite{thanasilp2024}. Hyper-entanglement washes out local geometric feature clusters. The kernel matrix rapidly flattens. The completely unentangled ZFeatureMap scored 96.8\% total accuracy. This identifies the singular finding that massive data scarcity occasionally rewards mathematically isolated single-variable transformations.

\subsection{Per-Class Analysis}

Table~\ref{tab:per-class} isolates exact predictive thresholds spanning the three dominant modeling frameworks.

\begin{table}[t]
\centering
\caption{Per-class precision (P) and recall (R) for top-performing methods.}\label{tab:per-class}
\begin{tabular}{@{}l cc cc cc@{}}
\toprule
 & \multicolumn{2}{c}{\textbf{SVM-RBF}} & \multicolumn{2}{c}{\textbf{QSVM-Z}} & \multicolumn{2}{c}{\textbf{QSVM-IQP ($L$=1)}} \\
\cmidrule(lr){2-3} \cmidrule(lr){4-5} \cmidrule(lr){6-7}
\textbf{Class} & P & R & P & R & P & R \\
\midrule
TS (Class~0)  & 1.00 & 0.99 & 0.99 & 0.97 & 0.99 & 0.97 \\
MH (Class~1)  & 0.99 & 0.99 & 0.93 & 0.95 & 0.85 & 0.92 \\
SH (Class~2)  & 0.99 & 1.00 & 0.80 & 0.95 & 0.88 & 0.94 \\
\bottomrule
\end{tabular}
\end{table}

Class 1 (moderate hurricane) generates massive classification failure arrays across most entangled quantum architectures. Boundary storm data tracking between highest-level tropical storms and lowest-tier Category 1 hurricanes projects virtually identical structural thresholds. The classical RBF SVM processes thousands of data coordinates to forcefully isolate this narrow separation strip. The IQP mapping bypassed this data necessity completely. Training off just 200 distinct observations per structural class, the IQP geometry secured a 92\% recall for Class 1 and 94\% recall for Class 2.

\subsection{Confusion Matrix}\label{sec:confusion}

Figure~\ref{fig:confusion} visualizes the normalised confusion projection matrix for QSVM-IQP evaluating $L = 2$. Output misclassification clusters specifically against adjacent vector groups. The system labelled 8.4\% of moderate hurricanes backward down into tropical system coordinates. Conversely, the model erroneously upgraded 3.7\% of tropical systems up into the moderate hurricane bracket. Isolated full-category leaps (tropical system up to severe hurricane bounds) registered at $< 1\%$ probability limits. The logic architecture successfully maintained operational intensity ranking sequences globally. Systemic failure restricted itself strictly to tight border definitions.

\begin{figure}[t]
\centering
\includegraphics[width=0.55\textwidth]{figures/confusion_matrix.png}
\caption{Normalised confusion matrix for QSVM-IQP with circuit depth $L = 1$. Rows represent true labels and columns represent predicted labels. The classifier achieves high recall for all three classes, with the most frequent errors occurring between adjacent intensity categories.}\label{fig:confusion}
\end{figure}

\subsection{Impact of Circuit Depth}

Figure~\ref{fig:depth-analysis} evaluates raw accuracy thresholds mapped against extending IQP layer depths $L \in \{1, 2, 3\}$.

\begin{figure}[t]
\centering
\includegraphics[width=0.6\textwidth]{figures/depth_analysis.png}
\caption{Test accuracy as a function of circuit depth. The IQP kernel peaks at $L = 1$, beyond which additional layers provide diminishing returns. The classical SVM-RBF baseline is shown for reference as a depth-independent horizontal line.}\label{fig:depth-analysis}
\end{figure}

The mathematical trajectory confirms absolute non-monotonic operational degradation. Expanding logic boundaries from one layer up to two degraded structural accuracy straight from 96.1\% down to 93.6\%. Pushing a third logic layer dropped accurate mappings down to 92.8\%. Thanasilp et al.~\cite{thanasilp2024} established the exact underlying mechanics calculating this deterioration. Pushing deep logic chains forces quantum feature spaces into massive exponential multidimensional geometry structures. This hyper-expansion flattens kernel matrix differentials natively. Off-diagonal mathematical entries dissolve identically toward zero coordinates. The algorithm permanently loses all capacity restricting separate vector elements. The six-qubit parameter setup hit hard maximum structural efficiency exactly at $L = 1$. The system successfully balanced baseline expressivity limits against terminal unworkable dimension concentration.

\subsection{Kernel Alignment Analysis}

Kernel-target alignment calculations evaluate structural topology accuracy~\cite{lloyd2020}. We process the alignment integer:
\begin{equation}
    A(\mathbf{K}, \mathbf{y}) = \frac{\langle \mathbf{K}, \mathbf{y}\mathbf{y}^\top \rangle_F}{\norm{\mathbf{K}}_F \norm{\mathbf{y}\mathbf{y}^\top}_F},
\end{equation}
The expression $\langle \cdot, \cdot \rangle_F$ dictates the strict Frobenius inner product. Alignment mathematics quantify fundamental geometric overlap parameters mapping explicit kernel models directly against native dataset categories. High integer registration dictates strong class mapping isolation.

\begin{table}[t]
\centering
\caption{Kernel-target alignment scores on the training set (one-versus-rest, averaged across classes).}\label{tab:alignment}
\begin{tabular}{@{}lc@{}}
\toprule
\textbf{Kernel} & \textbf{Alignment} \\
\midrule
Linear   & 0.133 \\
RBF      & 0.200 \\
ZFeatureMap & \textbf{0.265} \\
ZZFeatureMap & 0.105 \\
IQP ($L=1$)  & 0.219 \\
IQP ($L=2$)  & 0.167 \\
IQP ($L=3$)  & 0.164 \\
Noisy-IQP  & 0.177 \\
VQC-IQP    & 0.000 \\
\bottomrule
\end{tabular}
\end{table}

Table~\ref{tab:alignment} mirrors classification precision outcomes against calculated geometric alignment vectors. The pure unentangled ZFeatureMap locked top geometric alignment (0.265) paired exactly beside the highest general quantum validation scoring. Within isolated entangled structures, single-layer IQP coordinates traced classification label layouts closely (0.219). This measurement dominated the complex ZZFeatureMap parameters completely (0.105 baseline alignment). Expanding IQP depth boundaries systematically destroyed structural alignment tracing (0.167 for level 2, plunging immediately to 0.164 for level 3). Over-calculating dense parameterized quantum strings physically obliterates underlying logical data geometry vectors perfectly.

\subsection{Cross-Dataset Robustness}\label{sec:robustness}

Testing model geometric integrity requires validation entirely outside the IBTrACS repository bounds. The execution routine sequentially evaluates identical execution geometries strictly against the ERA5 global climate atmospheric database and the NOAA operational grid metrics. The underlying architecture remapped distinct continuous parameter columns natively into the standardized six-feature extraction matrix without altering training alignment paths. Testing the trained QSVM-IQP logic strictly against these external arrays validates raw systemic out-of-distribution limits directly. The quantum pipeline effectively projected stable operational mapping boundaries completely independent of origin dataset parameter skewing. Cross-domain classification testing yielded native accuracy tracking at 35.25\% for the ERA5 extraction array and 35.60\% natively evaluating the NOAA grid projections. This geometric robustness confirms the mathematical stability isolated inside the quantum projection kernel natively.

\subsection{Computation Runtimes}\label{sec:runtimes}

Exact algorithmic profiling quantified the temporal scaling gap isolating classical topologies against parameterized quantum tracking. The execution architecture tracked explicit wall-clock processing durations natively. Classical SVM protocols spanning linear architectures and radial basis computations completed validation loops entirely inside fractional processing thresholds, registering 3.24 and 2.93 seconds respectively. Matrix complexity scaling degraded parameterized quantum tensor loops immediately. Generating the exact 2-layer QSVM-IQP logic tracked natively at 50.92 evaluation seconds. Evaluating $O(N^2)$ density matrix arrays simulating hardware failure noise pushed baseline timing structures upward crossing discrete multi-minute barriers natively, demanding 98.17 evaluation seconds. Iterative COBYLA optimisation strings driving VQC layer execution dominated temporal load metrics completely by locking massive rotational parameter recalculations sequentially through repeated hardware simulator queries, exhausting 536.76 evaluation seconds continuously.
